%
% Author:  Jan Zabojnik
% E-mail:  jan.zabojnik@fel.cvut.cz
% Date:    03.02.2014 14:57
%
\documentclass[twoside]{article}
\usepackage[a4paper]{geometry}
\geometry{verbose,tmargin=2cm,bmargin=2cm,lmargin=2cm,rmargin=2cm}
\usepackage{fancyhdr}
\pagestyle{fancy}

% nastavení pisma a češtiny
\usepackage{lmodern}
\usepackage[T1]{fontenc}
\usepackage[utf8]{inputenc} %cp1250
\usepackage[czech]{babel}	

% odkazy
\usepackage{url}

% vícesloupcové tabulky
\usepackage{multirow}

%matika
\usepackage{mathtools}

%kody
\usepackage{listings}
\usepackage{color} %red, green, blue, yellow, cyan, magenta, black, white
\definecolor{mygreen}{RGB}{28,172,0} % color values Red, Green, Blue
\definecolor{mylilas}{RGB}{170,55,241}

\lstset{language=Matlab,%
    %basicstyle=\color{red},
    breaklines=true,%
    morekeywords={matlab2tikz},
    keywordstyle=\color{blue},%
    morekeywords=[2]{1}, keywordstyle=[2]{\color{black}},
    identifierstyle=\color{black},%
    stringstyle=\color{mylilas},
    commentstyle=\color{mygreen},%
    showstringspaces=false,%without this there will be a symbol in the places where there is a space
    numbers=left,%
    numberstyle={\tiny \color{black}},% size of the numbers
    numbersep=9pt, % this defines how far the numbers are from the text
    emph=[1]{for,end,break},emphstyle=[1]\color{red}, %some words to emphasise
    %emph=[2]{word1,word2}, emphstyle=[2]{style},    
}

%zarovnani rovnic
\usepackage{amssymb}
\usepackage{amsmath}

%seznamy
\usepackage{enumerate}

%tabulky
\usepackage{booktabs}

%stupne
\usepackage{textcomp}
\usepackage{gensymb}

% vnořené popisky obrázků
\usepackage{subcaption}

% automatická konverze EPS 
\usepackage{graphicx} 
\usepackage{epstopdf}

% odkazy a záložky
\usepackage[unicode=true, bookmarks=true,bookmarksnumbered=true,
bookmarksopen=false, breaklinks=false,pdfborder={0 0 0},
pdfpagemode=UseNone,backref=false,colorlinks=true] {hyperref}

% Poznámky při překladu
\usepackage{xkeyval}	% Inline todonotes
\usepackage[textsize = footnotesize]{todonotes}
\presetkeys{todonotes}{inline}{}

% Zacni sekci slovem ukol
%\renewcommand{\thesection}{Úkol \arabic{section}}
% enumerate zacina s pismenem
\renewcommand{\theenumi}{\alph{enumi}}

\begin{document}

% ========== Edit your name here
\author{Halodová L., Hodná J., Rouček T.}
\title{Semestrální práce B3M33ARO: 2D Barbie lokalizace}
\maketitle

% ---------------------------------
% ---------------------------------
% název sekce je generován automaticky jako: Úkol X
\section{Zadání}
Cílem semestrální práce je lokalizace panenky Barbie v neznámém prostředí s využitím robotu TurtleBot s RPLidarem a kamerou.
\section{Řešení}

\section{Originální nápady, které vylepší chování robota}
\subsection{Výběr vhodné cesty a jízda}
	% jak vybíráme cíl cesty a jak jedeme (průběžné vybírání cílů, hledání nových frontiers..., otáčení pokud nemáme frontiera)
\subsection{Filtr naplánované cesty}
	% Vybíráme 4. od začátku a nedojíždíme až na konec (-3)
\subsection{Výběr vhodných parametrů pro front-plan}
	Selekce vhodných parametrů pro nafukování detekovaných překážek v mapě...\\
	Jak můžete vidět v obrázku \ref{fig:front_plan}.
	\begin{figure}[!h]
     \centering
     \begin{subfigure}[b]{0.45\textwidth}
         \centering
         \includegraphics[width=\textwidth]{img/test.png}
         \caption{Test pic}
     \end{subfigure}
     \begin{subfigure}[b]{0.45\textwidth}
         \centering
         \includegraphics[width=\textwidth]{img/test.png}
         \caption{Test pic)}
         
     \end{subfigure}
        \caption{Test caption}
        \label{fig:front_plan}
	\end{figure}



\end{document}

